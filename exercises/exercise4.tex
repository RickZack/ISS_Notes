\chapter*{Exercises 4}
\section*{Short problems}

\section*{Security of IP networks}
\textbf{Q1 - Which protocol can be exploited for user authentication in dial-up, wireless, or virtual links (you can select multiple choices)?}
\begin{itemize}
    \item[A.] RADIUS
    \item[B.] EAP
    \item[C.] DIAMETER
    \item[D.] CHAP
\end{itemize}

\textbf{Q2 - At which level in the OSI stack is EAP (Extensible Authentication Protocol) exploited to support authentication?}
\begin{itemize}
    \item[A.] L2 (data link layer)
    \item[B.] L3 (network level)
    \item[C.] L7 (application level)
    \item[D.] L4 (transport level)
\end{itemize}

\textbf{Q3 - Assume that we have a small company which uses a NAS supporting EAP, PAP, CHAP. Look at the picture and tell which methods and protocol(s) can be used to authenticate a user (device) with the NAS to provide it access to Internet:}
\begin{itemize}
    \item[A.] RADIUS
    \item[B.] PAP
    \item[C.] CHAP
\end{itemize}
\begin{figure}[h]
    \centering
    \includegraphics[width = 0.70\textwidth]{exercises4/question3.png}
\end{figure}
  
\textbf{Q4 - Assume we have a small hotel that uses an Access Point (NAS) supporting EAP, CHAP. Take a look at the picture below and indicate which methods can be used to authenticate a user (device) with the AP to provide access to the Internet:}
\begin{itemize}
    \item[A.] EAP-TLS
    \item[B.] CHAP
    \item[C.] Username and password and EAP
    \item[D.] RADIUS
    \item[E.] None of the above
\end{itemize}
\begin{figure}[h]
    \centering
    \includegraphics[width = 0.70\textwidth]{exercises4/question4.png}
\end{figure}

\textbf{Q5 - EAP transports authentication messages between a user (supplicant) and a NAS:}
\begin{itemize}
    \item[A.] before the IP channel is established. Thus, EAP uses its own encapsulation protocol because IP packets are not available
    \item[B.] after the IP channel is established. Thus, EAP packets are encapsulated inside IP packets
    \item[C.] None of the above
\end{itemize}

\textbf{Q6 - EAP can transport authentication messages between a user and NAS (Network Access Server):}
\begin{itemize}
    \item[A.] only for PPP channels
    \item[B.] for any L2 protocol, such as PPP, 802.11 (Wi-Fi), 802.3 (Ethernet), 802.5 (token ring)
    \item[C.] only for 802.11 (Wi-Fi) and 802.3 (Ethernet)
\end{itemize}

\textbf{Q7 - What is an EAP method (e.g. EAP-TLS, EAP-TTLS, EAP-SRP, AKA-SIM)?}
\begin{itemize}
    \item[A.] it's a variant to allow EAP to be exploited in other (upper level) protocols
    \item[B.] it's the method used by EAP to perform user authentication
    \item[C.] it's the method used by EAP to transfer the messages of a specific (user) authentication protocol
    \item[D.] none of the above
\end{itemize}

\textbf{Q8 - RADIUS is a network authentication protocol that:}
\begin{itemize}
    \item[A.] is used for communication (client-server schema) between a user (supplicant) and a NAS
    \item[B.] is used for communication (client-server schema) between a NAS and a backend Authentication Server (AS)
    \item[C.] supports user authentication via PAP, CHAP, and EAP
    \item[D.] supports user authentication only via EAP
    \item[E.] may act as a proxy towards other AS
\end{itemize}

\textbf{Q9 - Assume we have a corporate network which has an Authentication Server (AS) in place, and uses an AP for network access control. Take a look at the picture below and indicate the protocol(s) typically used for the communication between AP and AS during authentication of supplicants:}
\begin{itemize}
    \item[A.] RADIUS
    \item[B.] PAP
    \item[C.] CHAP
\end{itemize}
\begin{figure}[h]
    \centering
    \includegraphics[width = 0.70\textwidth]{exercises4/question9.png}
\end{figure}

\textbf{Q10 - RADIUS implements the following security properties:}
\begin{itemize}
    \item[A.] authentication and integrity of requests, where the RADIUS requests are protected with a digital signature
    \item[B.] authentication and integrity of requests, where the RADIUS requests are protected with a keyed-digest
    \item[C.] authentication and integrity of responses, where the RADIUS responses are digitally signed by the RADIUS server
    \item[D.] authentication and integrity of responses, where the RADIUS responses are protected with a keyed-digest
    \item[E.] authentication of the server, through the calculation of an authenticator via keyed-digest
\end{itemize}

\textbf{Q11 - RADIUS implements the following security properties:}
\begin{itemize}
    \item[A.] confidentiality of the data in the RADIUS request by encrypting the data with an algorithm that can be negotiated
    \item[B.] confidentiality of the data in the RADIUS request by masking the (user) password (with an XOR operation)
    \item[C.] confidentiality of the data in the RADIUS response by encrypting the data with an algorithm that can be negotiated
    \item[D.] confidentiality of the data in the RADIUS response by masking the password (with an XOR operation)
    \item[E.] protection from replay of RADIUS response, by using a sequence number in the RADIUS response
    \item[F.] protection from replay of RADIUS response, by binding the response to a specific request via a Request Authenticator
\end{itemize}

\textbf{Q12 - Consider an architecture in which we have: a supplicant (user), a NAS, and a RADIUS server. If the
RADIUS server and the user have a shared secret (named “secret”), does this mean that a (malicious) user can create
a fake RADIUS response?}
\begin{itemize}
    \item[A.] yes, because the user knows the secret
    \item[B.] no, because in the construction of the RADIUS response it is used also a Request Authenticator generated by the NAS
    \item[C.] no, because NAS and the RADIUS server have also a “secret” that the user does not know (used in protecting the RADIUS response)
\end{itemize}

\textbf{Q13 - Assume a company uses RADIUS to authenticate users and Wi-Fi access-points and switches enabled with support
for 802.1x. What is the main role of the Wi-Fi access points and switches and what needs to be done to allow them to
work when new user authentication protocols appear (and are supported by EAP)?}
\begin{itemize}
    \item[A.] allow to perform authentication of the users connecting to the AP or switch by exploiting EAP
    \item[B.] encapsulate the EAP messages received from the users (supplicant) into RADIUS messages sent to the AS
    \item[C.] must be updated if new user authentication protocols appear and are supported by EAP
    \item[D.] no change is needed if new user authentication protocols appear and are supported by EAP
\end{itemize}


