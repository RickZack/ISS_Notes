\section{Security implementation in OSI levels}

\begin{figure}[h]
    \centering
    \includegraphics[page = 6,trim = 1cm 2.1cm 1cm 7cm, clip, width = 0.55\textwidth]{\slides}
\end{figure}

The main question is, “\textit{Which is the best OSI level to implement security?}” with many possibilities and answers. Typically, the “Presentation” layer is the only one in which security measures are not useful.

Unfortunately, \underline{there is not a single optimal level}. \textbf{The higher we go in the stack, the more specific our security functions can be}. For example, at the application level, it is possible to identify the user, commands, and data. Security functions are also independent of the underlying network, but if the functions are placed at the application level only, attacks at lower levels are possible (in particular, DoS attacks are available).

\textbf{The lower we go in the stack, the more quickly we can "expel" the intruders}, but the fewer data are available for the decision (e.g., only the MAC or IP addresses, no user identification, no commands).

In short, there is not an optimal level. You can decide whether to take one of these two risks or make a mixture by placing some security features at lower levels and then focus most of the security at the application level.


\subsection{DHCP (in)security}

When Layer 3 is reached, one of the first things activated is \textit{DHCP} because access to the network is given, and now the user needs to know the network parameters. DHCP is the protocol by which a device can ask to be assigned a valid network address. Unfortunately, the protocol is \textbf{non-authenticated} and a \textbf{broadcast protocol} that provides a response carrying \textit{IP address, netmask, default gateway, local nameserver,} and \textit{local DNS suffix}.

For this reason, the activation of a fake DHCP server is trivial because the DHCP request is a Layer 2 broadcast frame, and the only thing an attacker needs to do is to stay in the same broadcast domain as the victim and sniff the DHCP request.


Possible attacks from the fake DHCP are:

\begin{itemize}
    \item \textbf{Denial-of-service}:
          \begin{itemize}
              \item This can be done by providing a wrong network configuration.
          \end{itemize}

    \item \textbf{(Logical) Man-in-the-Middle (MITM)}:
          \begin{itemize}
              \item A valid IP address is provided to the victim, but it will be assigned a subnet with only the last two bits equal to zero. Therefore, only two addresses are valid: one of them is given to the user, and the other to the attacker as his default gateway. In that way, the attacked machine is isolated in a subnet of its own (logically, not physically). To communicate with all the nodes in the world, the victim has to send everything through the attacker.
              \item The replies could reach the original node without passing through the attacker. For this reason, it is possible to activate NAT, and it is possible to also intercept the replies.
          \end{itemize}

    \item \textbf{Malicious name-address translation}:
          \begin{itemize}
              \item The attacker declares himself as the local name server. Then, whenever the user needs to perform a name-to-address translation, the attacker will provide the wrong address. This is used, for example, for phishing and pharming.
          \end{itemize}
\end{itemize}

Various manufacturers have tried to provide some security improvement, such as switches (e.g., Cisco) that offer:

\begin{itemize}
    \item \textbf{DHCP Snooping}:
          \begin{itemize}
              \item Accepts only replies from "trusted ports."
          \end{itemize}

    \item \textbf{IP Guard}:
          \begin{itemize}
              \item Provides room only for IP addresses obtained from a valid DHCP server (but there is a limit on the number of recognized addresses).
          \end{itemize}
\end{itemize}

There is also RFC-3118 "Authentication for DHCP messages," which uses HMAC-MD5 to authenticate the messages, but it is rarely adopted because it is hardly configurable. Since HMAC is a symmetric protocol, it is needed to install a key on all the machines that need to use DHCP. This leads to the problem of key distribution. Furthermore, there is a problem of key management because if a key is captured, then it will be reusable, and since it is symmetric, any DHCP client could work as a DHCP server too.


\subsection{VPN}
\subsubsection{Security at Network Level (Layer 3)}

\begin{figure}[h]
    \centering
    \includegraphics[page = 10,trim = 2cm 2.2cm 2cm 10.5cm, clip, width = 0.55\textwidth]{\slides}
\end{figure}

Layer 3, which, given today's widespread use of the Internet, is predominantly IP, provides a crucial layer where meaningful security features can be implemented in a general sense. This is because it is the first layer that offers end-to-end connectivity. It allows for the creation of both \textbf{end-to-end protection} for Layer 3 homogeneous networks, such as IP networks, and \textbf{Virtual Private Networks (VPNs)}.

If it is possible to provide Layer 3 end-to-end protection, ensuring that data are secured as soon as they exit the server/client, it becomes irrelevant whether routers are properly managed or if the network being traversed is insecure. This is because the data are protected from the moment they exit the network interface until they reach the final network interface. Therefore, \ul{the only potential attacks are those originating from within the client/server}. Hence, security at Layer 3 allows us to disregard other attacks at the network level (except for Denial of Service - DoS).


\subsubsection{What is a VPN?}

\begin{figure}[h]
    \centering
    \includegraphics[page = 11,trim = 2cm 2.3cm 2cm 10cm, clip, width = 0.55\textwidth]{\slides}
\end{figure}


A \textit{Virtual Private Network} (VPN) is a technique, employing both hardware and/or software, that allows the creation of a private network while utilizing shared (or otherwise untrusted) channels and transmission devices. Instead of laying down their own cables and managing dedicated infrastructure, companies may prefer to establish a virtual segment of the network.

For instance, one could designate FIAT packets as "blue", allowing them to be exchanged only between blue endpoints. Similarly, packets from ENI could be labeled as "red" and permitted to be exchanged exclusively through red endpoints. While this conceptualization is beneficial, the devil lies in the details. Since it is impossible to visualize packets as blue or red, and the exact mechanism of their switching is uncertain, implementation details become crucial.

There are three techniques for creating a VPN:
\begin{itemize}
    \item \textbf{Private addressing}
    \item \textbf{Protected routing (IP tunnel)}
    \item \textbf{Cryptographic protection of network packets (secure IP tunnel)}
\end{itemize}


\paragraph*{VPN via Private Addresses}

In this basic VPN implementation, the networks included in the VPN use non-public addresses, making them unreachable from other networks (e.g., private IANA networks as listed in RFC-1918). Consequently, these networks are considered private, as they do not require authorization, and the packets in this case are not globally routable.

For example, a telecom provider wishing to share its infrastructure with various customers might allocate a distinct class of addresses to each customer. Access control lists (ACLs) on routers could then be implemented to ensure that packets are directed only to the allowed destinations.

However, this protection can be compromised under several circumstances:
\begin{itemize}
    \item \textbf{Guesses or discovers the addresses:} If the addresses are guessed or discovered, there is a risk of unauthorized access. If a class of addresses from another customer is found, an attacker could switch its own address to infiltrate that network.
    \item \textbf{Sniffing packets during transmission:} Since packets lack intrinsic protection, if network traffic can be sniffed, it may be possible to read the content of the packets.
    \item \textbf{Access to communication devices:} If someone gains access to the communication devices, they can read, change, or inject any type of packet.
\end{itemize}

There is minimal protection for packets, customers, and even for the infrastructure maintainer in this scenario. Consequently, the actual level of security is close to zero, despite the commercial availability of such services.


\paragraph{VPN via Tunnel}


This solution represents an improvement over the previous one. In this approach, routers encapsulate the entire Layer 3 packet as a payload within another packet, which can be \textbf{IP in IP}, \textbf{IP over MPLS}, or other techniques. Before encapsulation, border routers implement access control to the VPN via \textit{Access Control Lists} (\textbf{ACL}s). For instance, if a network belongs to the 10.1 address range, the destination can only be another network within the 10.1 range.

With this solution, providers gain protection against malicious end-users, as it prevents customers from changing the subnet to which they belong.
However, this protection can be circumvented by anyone managing a router or capable of sniffing packets during transmission. For instance, it does not provide protection to customers against attacks originating from within the geographical network (→ \ul{protection for providers but not for customers}).

If robust protection is desired, alternative techniques need to be considered.


\subparagraph{VPN via IP Tunnel}

\begin{figure}[h]
    \centering
    \includegraphics[page = 15,trim = 2cm 2.3cm 2cm 4cm, clip, width = 0.55\textwidth]{\slides}
    \caption{VPN via IP tunnel}
\end{figure}

Network 1 and Network 2 are depicted in the same color because they belong to the same subnet. When utilizing an IP tunnel, as the packets traverse from node A in subnet 1 to node B in subnet 2, they reach the border routers of subnet 1, which are responsible for encapsulation.

Router R1 identifies that B is in subnet 2, reachable through the border router R2. R1 then creates another packet that travels from R1 to R2, containing the original packet as its payload. The external IPv4 header of the tunnel is illustrated in the diagram. Upon receipt at router R2, the packet is decapsulated and forwarded to the final destination.

\textbf{Throughout transmission, the packet remains susceptible to being readable, manipulated, or injected}, signifying a lack of real security for the end user of the VPN.

The IP tunnel also presents a performance challenge: \textbf{fragmentation}. If the packet size equals the Maximum Transmission Unit (MTU), encapsulation will require fragmentation. In such instances, the maximum performance loss is 50\%, as two packets are generated instead of one. This impact is more pronounced for applications with large packets, typically non-interactive applications like file transfers. Consequently, this solution can also be a performance killer.


\paragraph{VPN via secure IP tunnel}

\begin{figure}[h]
    \centering
    \includegraphics[page = 17,trim = .4cm 4cm .5cm 5.5cm, clip, width = 0.55\textwidth]{\slides}
    \caption{VPN via secure IP tunnel}
    \label{fig:vpn-secure-ip}
\end{figure}

While the performance issue remains unresolved, the final solution offers enhanced security for end users. Prior to encapsulation, packets are protected with the following measures:
\begin{itemize}
    \item \textbf{MAC (Message Authentication Code):} Provides integrity and authentication.
    \item \textbf{Encryption:} Ensures confidentiality.
    \item \textbf{Numbering:} Guards against replay attacks.
\end{itemize}

No digital signature is employed due to its slowness, which would not align with the speed of current networks. If robust cryptographic algorithms are selected, the only viable attack is to disrupt communications (Denial of Service - DoS). This type of VPN is often referred to as \textbf{S-VPN} (\textit{Secure VPN}), representing the only VPN that can be considered secure (→ caution with VPNs promoted online).

In Figure \ref*{fig:vpn-secure-ip}, there is a router and a \textbf{TAP} (\textit{Tunnel Access Point}). Responsibilities are divided: the router oversees encapsulation/decapsulation, while the TAP is responsible for cryptographic protection. If this solution is implemented, and the TAP is managed by an external network provider, the security is compromised. Ideally, two separate devices should be in place: \ul{the client manages the TAP, and the ISP manages the router}.

\subsection{IPsec}

IPsec is the IETF architecture for Layer 3 security in IPv4/IPv6 designed to \textbf{create a Secure VPN (S-VPN) over untrusted networks} and \textbf{create end-to-end secure packet flows}. This is achieved through the definition of two specific packet types:

\begin{itemize}
    \item \textbf{AH (Authentication Header):}
          Provides integrity, authentication, and protection against replay attacks.
    \item \textbf{ESP (Encapsulating Security Payload):}
          Offers functions similar to AH, with the addition of payload confidentiality.
\end{itemize}

It is crucial to emphasize that \textbf{confidentiality can only be provided for the payload}; it is never possible to encrypt the header. Otherwise, intermediate systems would be unable to process the packets.

There is also a dedicated protocol for key exchange, named \textbf{IKE (Internet Key Exchange)}, to create and distribute keys in IP networks.

The IPsec security services include:

\begin{itemize}
    \item \textbf{Authentication of IP packets:}
          \begin{itemize}
              \item \textit{Data integrity}: The receiver can \textit{detect} if the packet has been manipulated. It is \underline{not} designed to prevent manipulation.
              \item \textit{Sender authentication}: A formal proof of the sender's identity. Note that this does not correspond to an IP address; IP addresses must not be trusted, as they can be completely fake (IP spoofing).
              \item \textit{(Partial) protection against "replay" attacks}: Challenges arise due to working at Layer 3, where packets can be lost or duplicated.
          \end{itemize}
    \item \textbf{Confidentiality of IP packets:}
          \begin{itemize}
              \item Data encryption (for the payload only).
          \end{itemize}
\end{itemize}


\subsubsection{IPsec Security Association (SA)}

\begin{figure}[h]
    \centering
    \includegraphics[page = 20,trim = 2cm 3cm 2cm 10cm, clip, width = 0.55\textwidth]{\slides}
    \caption{IPsec Security Association (SA)}
    \label{fig:ipsec-sa}
\end{figure}

\subsection{Security Associations (SA)}

These security features are associated with the concept of \textbf{Security Association (SA)}, which is a \textbf{unidirectional logical connection between two IPsec systems}. Each SA is associated with different security services. To achieve full protection for a bidirectional packet flow between two nodes, \textbf{two SAs are needed} (one from A to B and one for the packets from B to A).

In theory, it is possible to have different security features and different algorithms for the two directions, but normally, even if there are two distinct SAs, the same kind of protection/algorithms are used.
Imagine that the sender (node A) is transmitting data (which may require confidentiality), but the response from B is a very simple thing (e.g., "received" or "bad") that does not need confidentiality. In this case, it is possible to avoid the encryption of the returning packet.

Security Associations are managed through two local databases, \textbf{which are not real databases}
(i.e., no implementation of servers like SQL, Oracle; it means that it is just \ul{a collection of data}):


\begin{itemize}
    \item \textbf{SPD (Security Policy Database):}
          \begin{itemize}
              \item It contains a list of security policies to apply to different packet flows.
              \item A-priori configured (e.g., manually) or connected to an automatic system (e.g., ISPS, which stands for Internet Security Policy System).
          \end{itemize}
    \item \textbf{SAD (SA Database):}
          \begin{itemize}
              \item It is a runtime database that contains the list of active SAs and their characteristics (e.g., algorithms, keys, parameters) to create protected traffic for that specific SA.
          \end{itemize}
\end{itemize}

\begin{figure}[h]
    \centering
    \includegraphics[page = 22,trim = 3cm 2.5cm 2cm 4.5cm, clip, width = 0.55\textwidth]{\slides}
    \caption{Local operations performed by a IPsec module when sending a packet}
\end{figure}

Suppose to be at a sending node within the TCP/IP stack. An IP packet has been created to be sent at L2, but on this node there's IPsec. When the packet is ready to be sent, the IPsec module starts working. The first question is: \textit{which policy should be applied for this packet?}

The answer is provided by the \textbf{SPD}. It can be \textit{"you should apply these security rules"} or \textit{"you don't need any kind of protection for this packet; go straight to L2"}.
If protection is needed and this is the first packet of this specific network flow, IPsec proceeds in creating a Security Association. Otherwise, there is already an existing SA, and it proceeds in reading the parameters associated with that SA by consulting the \textbf{SAD}. This will provide it with the algorithms and parameters to enrich the packet, and finally, the IP packet will be protected with IPsec and sent to L2 for the actual transmission.



\subsubsection{Transport mode IPsec}
\begin{figure}[h]
    \centering
    \includegraphics[page = 23,trim = 1cm 3cm 3cm 11cm, clip, width = 0.55\textwidth]{\slides}
    \caption{IPsec used for end-to-end security in transport mode}
\end{figure}

Transport Mode IPsec is used for \textbf{end-to-end security}, primarily employed by hosts rather than gateways (with the exception of traffic for the gateway itself, e.g., SNMP, ICMP). The original packet is split into two parts, and a new header is inserted between the IPv4 header and the TCP/UDP header. Thus, the IPv4 header will indicate that it is transporting IPsec (instead of TCP/UDP). Inside the IPsec header, there will be another field specifying the actual payload being transported.
\begin{itemize}
    \item \textbf{Pro:} It is computationally light.
    \item \textbf{Con:} No protection of header variable fields.
\end{itemize}


\subsubsection{Tunnel mode IPsec}
\begin{figure}[h]
    \centering
    \includegraphics[page = 24,trim = 1cm 3cm 2cm 9cm, clip, width = 0.55\textwidth]{\slides}
    \caption{IPsec used for end-to-end security in transport mode}
\end{figure}

Tunnel Mode IPsec is utilized to \textbf{create a VPN}, typically among \underline{gateways}. It is not created among routers; the correct term is gateway, which serves as a contact point between a network assumed to be secure and a network that is not secure. The gateway enhances protection by establishing the secure IPsec tunnel. It takes the original packet with the end-to-end header and encapsulates it inside a tunnel. This tunnel is initiated by the sending gateway and has the destination gateway protecting the network where the destination is located. The sending gateway applies IPsec to this IP in IP packet.

\begin{itemize}
    \item \textbf{Pro:} Complete protection of the packet, including header variable fields.
    \item \textbf{Con:} Computationally heavy.
\end{itemize}

Although not as common, IPsec in tunnel mode can also be used for end-to-end communication, even though it is typically adopted between border elements of larger networks. This setup is also known as a \textbf{site-to-site VPN} since the entities involved are typically entire networks.


\subsubsection{Authentication Header (AH)}
\begin{figure}[h]
    \centering
    \includegraphics[page = 26,trim = 1cm 6cm 1cm 5cm, clip, width = 0.55\textwidth]{\slides}
    \caption{IPsec AH header format according to RFC-4302}
\end{figure}

AH stands for \textbf{Authentication Header}. There have been three major versions of IPsec.

\subparagraph{Mechanism of the \textbf{first version} (RFC-1826):}
\begin{itemize}
    \item Data integrity and sender authentication
    \item Compulsory support of keyed-MD5 (RFC-1828)
    \item Optional support of keyed-SHA-1 (RFC-1852)
\end{itemize}

\subparagraph{Mechanism of the \textbf{second version} (RFC-2402):}
\begin{itemize}
    \item Data integrity, sender authentication, and (partial) protection from replay attacks
    \item HMAC-MD5-96
    \item HMAC-SHA-1-96
\end{itemize}

\subparagraph{Header format of the third version (RFC-4302)}
The format of the header added to the IPsec packet includes:
\begin{itemize}
    \item \texttt{Next header} field because this is a pseudo-protocol. In the IP header, it will indicate that it is transporting AH. Inside the AH, there is the real transporting packet field.
    \item \texttt{Length} parameter of 1 byte to describe the length of the packet.
    \item \texttt{Reserved} bytes for future uses.
    \item \texttt{Security Parameters Index (SPI)}: 32 bits to refer in a quick and easy way to all the parameters needed to verify in the packet.
    \item \texttt{Sequence number} to avoid replay attacks.
    \item \texttt{Integrity Check Value (ICV)}: Variable number of 4-byte words to provide authentication data (digest).
\end{itemize}



\begin{figure}[h]
    \centering
    \includegraphics[page = 27,trim = 1cm 2.1cm 1cm 2cm, clip, width = 0.90\textwidth]{\slides}
    \caption{The processing of a received IPsec packet}
    \label{fig:IPsec-receiving-packet}
\end{figure}

\paragraph{Receiving IPsec packet}
As shown in Figure \ref*{fig:IPsec-receiving-packet}, when an IPsec packet is received, it is protected with AH. The process begins with the \textbf{extraction} of the \texttt{AH}, and from it, the  \texttt{ICV} is extracted — the \texttt{received authentication value} (the digest computed by the sender).
Subsequently, the received packet undergoes \textbf{normalization}, which involves putting the packet in the same condition as it was at the sender to compute the same kind of hash.

Once the \texttt{normalized IP packet} is available, it is necessary to \textbf{compute the authentication value (ICV)}. For this purpose, the \textit{Security Parameter Index (SPI)} is used inside the \textit{Database of the Security Association (SAD)}. It serves as a pointer indicating the algorithm and parameters to be used. These parameters are employed to compute the authentication value, and then it is checked whether the two values (the one computed and the one received from the sender) are the same.

If the two values are equal, then the sender is authenticated, and the packet is integral. However, if the two values are not equal, there could be a fake sender and/or manipulated packet.



\subparagraph{Normalization for AH}
