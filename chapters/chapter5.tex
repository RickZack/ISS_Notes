\chapter{Firewall}

\section{What is a firewall}
A firewall is a \textbf{"wall to protect against fire propagation"}: this concept originates from regulations common in countries where houses are primarily constructed with wood. According to these regulations, a brick wall must be inserted between two houses under specific conditions to prevent the spread of fire. Similarly, in the realm of information systems security, a firewall is positioned to prevent the easy propagation of an attack from one network to another in the event of a network being compromised (akin to being "on fire"). This is particularly relevant when two networks have varying security levels, with one considered untrusted, and an attack on it could potentially spread to a more secure network.

Therefore, in addition to \ul{placing firewalls at the boundaries of networks with different security levels}, it is crucial to consider the \textbf{directionality} of the \textbf{controlled flow}.


An important aspect to consider is the directionality of the flow being controlled:
\begin{itemize}
    \item \textbf{Ingress Firewall}:
          \begin{itemize}
              \item It is a filter that checks \textbf{incoming connections} from the untrusted network.
              \item The purpose is typically to select (public) services offered; it limits which internal services are offered to the external network.
              \item The problem is that sometimes the request to open a connection is part of an application exchange initiated by internal users.
          \end{itemize}
    \item \textbf{Egress Firewall}:
          \begin{itemize}
              \item It is a filter that checks \textbf{outgoing connections}.
              \item The purpose is typically to check the activity of the personnel. This can include control of websites visited, verifying that confidential documents are not sent outside the network, or ensuring that a user is not downloading potentially dangerous files.
          \end{itemize}
\end{itemize}



This classification is straightforward for channel-based services (e.g., TCP applications) but challenging for message-based services (e.g., ICMP, UDP applications). In such cases, the firewall is typically bidirectional, as a clear distinction between ingress and egress would be meaningless.


\section{Firewall design}


\subsection{The Three Commandments of Firewall}
When designing a firewall, the primary consideration is the desired security level. However, there exists an inverse relationship between the level of protection and the functionalities offered. \textbf{The higher the protection level, the fewer functionalities can be provided}.
Consequently, all solutions are compromises between these two objectives.

Moreover, when designing a firewall, one should adhere to the principles articulated by the pioneers of the firewall (D. Cheswick and S. Bellovin):


\begin{itemize}
    \item [I.] \textbf{The firewall must be the only contact point of the internal network with the external one}.
    \item [II.] \textbf{Only the "authorized" traffic can traverse the firewall}. This implies having a precise list of the allowed traffic, including protocols, ports, or external nodes. Failure to do so may result in either blocking valid traffic (thus limiting functionalities) or being too permissive and exposing the network to potential attacks.
    \item [III.] \textbf{The firewall must be a highly secure system itself}. Many companies use a powerful general-purpose computer for the firewall and gradually load additional components (e.g., web server, database server) onto it. This practice should be avoided, as adding more software increases the likelihood of introducing bugs that could be exploited in an attack.
\end{itemize}

\subsubsection{Authorization policies}
For the second principle, when designing a firewall, authorized traffic must be identified. In expressing the authorization policy, there are two possible strategies:
\begin{itemize}
    \item \textbf{Whitelisting: "All that is not explicitly permitted is forbidden"}
          \begin{itemize}
              \item The firewall opens only a few kinds of communication, leading to \underline{higher security}.
              \item \underline{More difficult to manage}, especially if users were accustomed to free connectivity. This could require providing explanations about things that are no longer permitted.
              \item \underline{This is the recommended strategy}.
          \end{itemize}
    \item \textbf{Blacklisting: "All that is not explicitly forbidden is permitted"}
          \begin{itemize}
              \item Implies studying what can be a security problem and forbidding that kind of traffic, typically leading to \underline{lower security} (open gates). It is like saying that the vulnerability is kept until it is discovered.
              \item \underline{Easier to manage}.
          \end{itemize}
\end{itemize}



\section{Basic components of a firewall}
As mentioned earlier, a firewall comprises several components:
\begin{itemize}
    \item \textbf{Screening router (choke):} A router that filters traffic at the network level.
    \item \textbf{Bastion host:} A secure system that undergoes periodic auditing. It serves as the first line of defense against attacks.
    \item \textbf{Application gateway (proxy):} A service that operates on behalf of an application, facilitating communication outside the network with access control.
    \item \textbf{Dual-homed gateway:} A system with two network cards, acting as a bridge between two different networks. Normal routing is disabled, and components on top decide whether to forward or block traffic.
\end{itemize}

Depending on the network stack layer considered by a firewall, different terms are used to refer to it:
\begin{itemize}
    \item \textbf{Packet filter:} Analyzing traffic at the network level, exploiting information contained in that level.
    \item \textbf{Circuit gateway:} Operating at layer 4, considering TCP streams or UDP datagrams as units for filtering.
    \item \textbf{Application gateway:} Examining application data in detail.
\end{itemize}

Usually, the higher we go in the stack, the more accurate the detection of attacks becomes, but the slower the process.

\begin{figure}[h]
    \centering
    \includegraphics[page = 7,trim = .5cm 2.5cm .5cm 4cm, clip, width = 0.70\textwidth]{\slides}
    \caption{A which level the controls are made?}
\end{figure}


\subsection{Firewall technologies}
Firewall Technologies
Depending on the network level at which controls are performed, different terms are used:

\begin{itemize}
    \item (Static) packet filter
    \item Stateful (dynamic) packet filter
    \item Cut-off proxy
    \item Circuit-level gateway/proxy
    \item Application-level gateway/proxy
    \item Stateful inspection
\end{itemize}
Differences are observed in terms of:
\begin{itemize}
    \item Controls to be performed (= threats detected):
          \begin{itemize}
              \item They are related to the threats that can be detected.
          \end{itemize}
    \item Performance:
          \begin{itemize}
              \item Understanding the different performance levels provided by various technologies is important, typically because security comes with a price.
          \end{itemize}
    \item Protection of the firewall O.S.:
          \begin{itemize}
              \item Various solutions offer different levels of protection for the firewall itself.
          \end{itemize}
    \item Keeping or breaking the \textit{client-server model}:
          \begin{itemize}
              \item The discussed firewalls are placed between the client and the server. It could be transparent, meaning that if the traffic is permitted, the client can directly communicate with the end server. Alternatively, the firewall can act as a proxy, allowing the client to communicate with the firewall, which then communicates with the server. The second model, where the firewall acts as a proxy, is more secure, but the firewall itself becomes an element prone to attacks.
          \end{itemize}
\end{itemize}


\subsubsection{(Static) Packet filter}
Originally, it was available on routers and performed packet inspection at the network level by checking just the IP header and, when available, also the transport header. The reason for this is that once a packet is received, it should be forwarded or discarded in the minimum possible time.

This kind of solution has pros and cons:

\begin{itemize}
    \item \textbf{It is independent of the kind of applications}
          \begin{itemize}
              \item Good scalability (if throughput increases).
              \item Approximate controls: easy to "fool" (e.g., IP spoofing, fragmented packets).
          \end{itemize}
    \item \textbf{Good performance}
    \item \textbf{Low cost} (available on routers and in many OS).
    \item \textbf{Difficult to support services with dynamically allocated ports} (e.g., FTP).
    \item \textbf{Complex to configure}: all the rules are expressed in terms of IPs, port, and protocols.
    \item \textbf{Difficult to perform user authentication}: it comes from the fact that it uses only information available at L3.
\end{itemize}


% TODO: remove this, seems like it's no longer in the slides
\subsubsection{Stateful (dynamic) packet filter [REMOVED?]}

\begin{itemize}
    \item \textbf{Conceptually similar to packet filter but "state-aware":} it means that it remembers the previous packets in order to be faster in processing the new ones. It can look inside packets for commands that define the ports that will be used (e.g., FTP PORT command).
    \item \textbf{It can distinguish new connections from those already open:}
          \begin{itemize}
              \item Keeps state tables for open connections;
              \item Packets matching one row in the table are forwarded without any further control (fast transmission).
          \end{itemize}
    \item \textbf{Better performance than a packet filter:}
          \begin{itemize}
              \item SMP (Symmetrical Multi-Processing) support can be enabled, thanks to the use of state tables.
          \end{itemize}
    \item \textbf{Still has many of the static packet filter limitations}.
\end{itemize}

\subsubsection{Application-level gateway}
It is composed of a set of proxies that inspect the packet payload at the application level. The gateway often \textbf{requires modifications to the client application} and may optionally \textbf{mask/renumber the internal IP addresses}. When used as part of a firewall, it usually also performs \textbf{peer authentication} (typically for egress firewall). Since the proxy understands the application-level commands, it can provide \textbf{top security}, for example, against buffer overflow attacks on the target application. There are differences between forward-proxy (egress) and reverse-proxy (ingress).

In general, when working at the application level, there may be rules that are \textbf{more fine-grained and simpler} than those of a packet filter because it is possible to express rules in terms of application-level commands and data. However, every application needs a specific proxy because it will have its own commands and data. This implies:

\begin{itemize}
    \item Delay in supporting new applications.
    \item Heavy reliance on computational resources: a separate process is needed for each connection, typically running in user mode.
    \item Low performance (because they are user-mode processes).
\end{itemize}

It is possible to use SMP that may improve performance. However, this completely breaks the client/server model, which means:

\begin{itemize}
    \item More protection for the server.
    \item Ability to authenticate the client.
    \item Not transparent to the client: the application needs to be configured to use the proxy.
\end{itemize}

Since it is not transparent, the OS of the firewall may be exposed to attacks because it needs to process all the packets. Additionally, there is also a problem with application-level security techniques (e.g., SSL) that will not allow inspecting the traffic. For this reason, there are some variants:

\begin{itemize}
    \item \textbf{Transparent Proxy:}
          \begin{itemize}
              \item Less intrusive for the client.
              \item Requires more effort (packet rerouting + destination extraction).
          \end{itemize}

    \item \textbf{Strong Application Proxy} (checking semantics, not just syntax):
          \begin{itemize}
              \item Only some commands/data are forwarded. For example, in the case of the HTTP protocol, this could allow only GET and POST commands to be allowed.
              \item This is the only correct configuration for a proxy\footnote{according to Lioy.}.
          \end{itemize}
\end{itemize}



\subsubsection{Circuit-level Gateway}

Between the packet filter and the application-level gateway, there is also the \textbf{circuit-level gateway}. It is a generic proxy (which is not "application-aware") that \textbf{creates a transport-level circuit} between the client and server, but it does not understand or manipulate the payload data in any way. It simply copies TCP segments or UDP datagrams (if they match the access control rules), but in doing this, it \ul{re-assembles the IP packets, providing protection against some L3/L4 attacks}.

For the server, all attacks related to the TCP handshake are no longer possible because there is no TCP handshake between the client and server. \ul{The client performs the TCP handshake with the gateway, and then the gateway performs a correct handshake with the server}.

In conclusion, it breaks the TCP/UDP-level client/server model during the connection and provides:

\begin{itemize}
    \item \textbf{More Protection for the Server:}
          \begin{itemize}
              \item Isolated from all attacks related to the TCP handshake.
              \item Isolated from all attacks related to IP fragmentation; an attacker could fragment a packet with which they perform the attack, and such an attack will not be detected by a simple packet filter.
          \end{itemize}
    \item \textbf{May Authenticate the Client:}
          \begin{itemize}
              \item But this requires modification to the application.
          \end{itemize}
\end{itemize}

However, it still exhibits many limitations of the packet filter. A well-known example is \textbf{SOCKS}.



\subsubsection{HTTP (forward) proxy}
\begin{figure}[h]
    \centering
    \includegraphics[page = 16,trim = 1cm 3.5cm 1cm 15cm, clip, width = 0.55\textwidth]{\slides}
\end{figure}

A forward proxy, typically HTTP, acts as an HTTP server, functioning as a front-end and passing requests to the real external server. In addition to network \textit{Access Control Lists} (\textbf{ACL}), the benefits include:

\begin{itemize}
    \item Shared cache of external pages for all internal users.
    \item Authentication and authorization of internal users.
    \item Various controls (e.g., allowed sites, transfer direction, data types, …).
\end{itemize}

It is a typical component of the \textbf{egress} firewall. By contrast, the \textit{reverse proxy} is used for the \textbf{ingress firewall}.

\subsubsection{HTTP reverse proxy}

An HTTP server acts as a front-end for the real server(s) to which the requests are passed. It can implement ACL again in case it is possible to limit clients, but typically, an ingress firewall does not limit clients that can contact the server. It is possible to perform \textbf{content inspection}.
Benefits of this proxy include:

\begin{itemize}
    \item \textbf{Obfuscation} (no info about the real server):
          Since the server proxy is responding to the client, it can declare that it is a generic proxy without providing any information to the real software that implements the final server.
    \item \textbf{SSL accelerator} (with unprotected back-end connections …):\\
          Reverse proxy can be the \underline{endpoint for SSL/TLS}. In that sense, it is an SSL accelerator because if the channel is terminated here, it is possible to place an HSM and then get the additional benefit that the traffic comes in clear after the TLS channel is terminated. This permits performing content inspection.
    \item \textbf{Load balancer}
    \item \textbf{Web accelerator} (=cache for static content)
    \item \textbf{Compression}
    \item \textbf{Spoon feeding}:\\
          Gets from the server a whole dynamic page and feeds it to the client according to its speed, thus unloading the application server.
\end{itemize}


\begin{figure}[h]
    \centering
    \includegraphics[page = 18,trim = 1cm 3cm .5cm 4cm, clip, width = 0.55\textwidth]{\slides}
    \caption{Reverse proxy: possible configurations}
    \label{fig:reverse-proxy}
\end{figure}


There are two possible configurations for a reverse proxy (Figure \ref{fig:reverse-proxy}).
\begin{itemize}
    \item [left.] The best solution is that, if conceptually a \textit{three-legged firewall} (see Chapter \ref{chap:screened-subnet-v2}) is used, \textbf{the reverse proxy should be placed on the DMZ because it will be the public interface}. Behind it, the equivalent servers (which are the application servers being accessed by external users) can be placed. The problem arises when the application server needs access to data located inside the internal network. In this case, the server should pass back through the proxy, traverse the firewall, and finally enter the internal network.
    \item [right.] To handle this case and avoid the issues just explained, an alternative solution is also possible: \textbf{the reverse proxy is on the DMZ, and then a VPN connection is established between the reverse proxy and the servers, which are in the internal network}. This approach should limit the risks because there is no direct access to those servers, only through the reverse proxy. However, the first solution is the suggested one, as in that case, an attack against the (public) servers is confined to the DMZ.
\end{itemize}


\subsubsection{WAF (Web Application Firewall)}

Since web applications are widely used nowadays, there is an increasing number of threats. WAF is a module installed at a proxy (forward and/or reverse) to \textbf{filter the application traffic}. It checks:

\begin{itemize}
    \item HTTP commands
    \item Header of HTTP request/response
    \item Content of HTTP request/response
\end{itemize}

The most widely known and used WAF is \textbf{ModSecurity} (opensource project), which is a plugin for Apache and NGINX (representing 5\% and 30\% of worldwide HTTP servers). ModSecurity is so popular that OWASP (Open Web Application Security Project) has developed a specific \textbf{ModSecurity Core Rule Set} \textbf{(CRS)}. This means that \ul{OWASP studied the most frequent and well-known attacks and wrote a set of rules that permit ModSecurity to detect and drop those attacks}.


% TODO: consider renaming/removing this section, or place it somewhere else
\section{Architectures}

\subsection{"Packet filter" architecture}

\begin{figure}[h]
    \centering
    \includegraphics[page = 20,trim = 1cm 2.5cm 1cm 4cm, clip, width = 0.55\textwidth]{\slides}
    \caption{"Packet filter" architecture}
\end{figure}

The core concept is to equip the switching element with filtering capabilities. Typically, since this is a Layer 3 (L3) network device, the kind of filtering it can perform is at the IP and upper levels, implementing a packet filter. The advantage of this architecture is that there is no need for dedicated hardware. Moreover, since this filter does not address the application level in any way, there is no need for a proxy, and hence, there is no need to modify the applications. It is a simple, easy, cheap, and... \textit{insecure} solution since the kind of controls it can make are very trivial, usually based on protocols, ports, and addresses. One final drawback is that the router is a single point of failure, meaning that a bug could allow the attacker to bypass the control and access the internal network.

\subsection{"Dual-homed gateway" architecture}

\begin{figure}[h]
    \centering
    \includegraphics[page = 22,trim = 1cm 2.5cm 1cm 4cm, clip, width = 0.55\textwidth]{\slides}
    \caption{"Dual-homed gateway" architecture}
\end{figure}

Because relying solely on filtering at the network layer does not provide high security, another element is needed to improve the architecture, performing inspections at higher levels. In this architecture, the traffic permitted by the packet filter will not directly enter the internal network. Instead, it will be further checked by a second element, generally referred to as a "gateway". This gateway is actually "dual-homed" because it has two network cards, with routing disabled. A packet coming from one interface goes to the other if it respects the imposed rules.

This architecture is still easy to implement because two different components deal with separated aspects. It requires small additional hardware requirements; just a general-purpose machine with the proper gateway software installed will suffice. Moreover, \ul{since the gateway is the frontend for the entire internal network, it can masquerade the internal addresses}, even without using NAT.

The big problem with this solution is its rather \underline{inflexibility}. Even if traffic is permitted in the packet filter, it must also pass the check at the gateway. Taking electronic mail as an example, typically, in incoming mail, it is difficult to limit the sender (other than forbidding well-known sites that send spam). Therefore, it is checked at the mail server, which is internal to the network. The gateway would reconstruct the application-level packet just to see that it is an email, and it is not its job to inspect it. Consequently, it will send it to the internal network. \ul{The inflexibility lies in the fact that a packet is double-checked even if the second check is useless because nothing can be decided with the information that the gateway has}. The actions performed by the gateway lead to a large work overhead. One thing to note is that this solution implements a \textbf{double line of defense}, applying the "\textit{defense in depth}" principle.


\subsection{"Screened host" architecture}

\begin{figure}[h]
    \centering
    \includegraphics[page = 24,trim = 1cm 2.5cm 1cm 4cm, clip, width = 0.55\textwidth]{\slides}
    \caption{"Screened host" architecture}
\end{figure}


Improving the "dual-homed gateway" architecture means avoiding the bottleneck at the gateway. In the "screened host" architecture, the packet filter is also connected to the internal network, but there is also the gateway, which is connected to the packet filter. This arrangement allows packets that do not need a double-check to be directly forwarded to the internal network. However, it reintroduces the problem of a \underline{single point of failure} in the packet filter, and a \ul{double line of defense is implemented only for those packets that need to be processed more in-depth}.

To summarize:
\begin{itemize}
    \item The \textit{router}:
          \begin{itemize}
              \item Blocks traffic from internal to external unless it comes from the bastion.
              \item Blocks traffic from external to internal unless it goes to the bastion.
              \item Exception: directly enabled services.
          \end{itemize}
    \item The \textit{bastion host} runs a circuit/application gateway to control authorized services.
    \item This architecture is more expensive and complex to manage because the work of the two systems must be synchronized.
    \item It is more flexible (skip control over some services/hosts).
    \item Only the hosts/protocols passing through the bastion can be masked (unless the Packet Filter (PF) uses NAT).
\end{itemize}


\subsection{"Screened Subnet" Architecture}\label{chap:screened-subnet}

\begin{figure}[h]
    \centering
    \includegraphics[page = 26,trim = 1cm 2.5cm 1cm 4cm, clip, width = 0.55\textwidth]{\slides}
    \caption{"Screened subnet" architecture}
    \label{fig:screened-subnet}
\end{figure}

The main problem introduced with the "screened host" architecture is the single point of failure in the packet filter. In the "screened subnet" architecture, the packet filter is split because it conceptually performs two different operations:

\begin{itemize}
    \item Deciding if incoming traffic should be permitted or denied.
    \item Deciding if traffic should be forwarded to the internal network.
\end{itemize}

In this solution, one packet filter is connected to the external network, while the other one acts like a bridge from the "intermediate" network to the internal one. First, an incoming packet is checked. If it is permitted, it is forwarded to the second packet filter that will still verify if the rule is valid and then forward it to the internal network. This way, there is a \ul{double defense, even for packets that will be directly transmitted to the internal network}. For packets that need further inspections, there will be three defense lines, as the second packet filter will send them to the bastion. Please note that, to have a proper double line of defense implemented by the two routers, they should be sourced from different vendors. If they are equal, the same bugs may affect security.

In this scheme (Figure \ref{fig:screened-subnet}), there is \textbf{a network that is completely decoupled from both the external and the internal network} (highlighted in red in the figure): this is called the \textbf{DMZ} (\textit{De-Militarized Zone}).
Typically, the DMZ is not only home to the gateway but also to other hosts, such as public servers (e.g., web servers or remote access systems). This implies that public servers should be placed in the DMZ. This solution is more expensive than the others because it is more complex.

\subsubsection{"Screened Subnet" Architecture (Version 2)}\label{chap:screened-subnet-v2}

\begin{figure}[h]
    \centering
    \includegraphics[page = 28,trim = 1cm 2.5cm 1cm 9cm, clip, width = 0.55\textwidth]{\slides}
    \caption{"Screened subnet" architecture v2}
    \label{fig:screened-subnet-v2}
\end{figure}

To reduce costs and simplify management, often the routers are omitted, and their function is incorporated into the gateway: this solution is also called the "\textbf{three-legged firewall}". In this solution, the firewall is a single element, typically a general-purpose computer, equipped with \textbf{three network cards}: one connects to the \underline{external network}, one to the \underline{internal}, and the third one implements the \underline{DMZ}.

The components of the previous solution are software modules inside a single element. From a functional point of view, there are the same functionalities, but the solution is intrinsically less secure because:

\begin{itemize}
    \item There is again a \textit{single point} of failure, the single machine hosting all the processes.
    \item The \textit{complexity} of this node makes room for attacks due to implementation or configuration errors.
\end{itemize}

Despite these considerations, many companies tend to implement this solution because it provides numerous functionalities while keeping costs limited, and the management is simple due to having only one interface for overseeing everything. Some companies claim to have implemented a \textit{"four-legged firewall"} or a \textit{"five-legged firewall"}: the number of legs comes from having multiple interfaces toward the internal network or from deploying multiple DMZs. Each one can be assigned to a different department of the company, mostly for administrative decisions.


\section{Local/personal firewall}

Until now, we discussed the network firewall, which is a filter situated between two networks (internal and external). However, this kind of firewall faces a problem with HTTPS, as channels are encrypted, making it difficult to inspect packets.

This has led to moving the firewall to a location where traffic can be inspected. In HTTPS, traffic is encrypted on the client and server sides. This implies that the firewall should be moved to the client or server: a \textbf{local firewall} if installed on the server or a \textbf{personal firewall} if installed on the client. It is \textbf{directly installed at the node to be protected}, safeguarding a single node instead of a whole network. It is \textbf{typically a packet filter} but, concerning a normal network firewall, it may \textbf{limit the processes} that are allowed to:
\begin{itemize}
    \item Open network channels towards other nodes (i.e., act as a client)
    \item Answer network requests (i.e., act as a server)
\end{itemize}

For example, this firewall can limit traffic going to an unknown process because it might be malware on the client attempting to gather information.

It is crucial to limit the spread of malware and trojans or detect plain configuration mistakes (for example, some services installed by default on the operating system that send or receive data). Firewall management must be separated from system management; otherwise, the people installing a service may also activate the firewall to permit that kind of traffic (a decision that should be made by the security manager).



\section{Protection offered by a firewall}

The analogy presented here illustrates that in the areas of the wall where bricks are removed to allow traffic, that is the part where fire can enter the system. For all channels enabled through the firewall, additional protections are necessary:

\begin{itemize}
    \item VPN
    \item "Semantic" firewall / IDS (Intrusion Detection System)
    \item Application-level security
\end{itemize}


\section{Intrusion Detection System (IDS)}

\textbf{IDS} is another critical component of contemporary security solutions; it is a \textbf{system designed to identify actors using a computer or a network without authorization}.
Depending on the features, it can be extended to \textbf{identify authorized actors violating their privileges}.

IDS is based on a crucial hypothesis: \textbf{the behavioral "pattern" of non-authorized users differs from that of authorized ones}.

An IDS can base its behavior on two strategies.

\begin{itemize}
    \item \textbf{Passive IDS}:\\
          Tries to identify visible signs of an attack, using:
          \begin{itemize}
              \item Cryptographic checksum (e.g., tripwire) checking for modified files on the server that should not be modified;
              \item Pattern matching (“attack signature”) looking for specific packets which are typical of an attack;
          \end{itemize}

    \item \textbf{Active IDS}:\\
          Tries to identify unknown attacks or known attacks but before they produce their negative result. It goes through three steps:
          \begin{itemize}
              \item “Learning” = accurate statistical analysis of the system behavior;
              \item “Monitoring” = active statistical info collection of traffic, data, sequences, actions;
              \item “Reaction” = comparison against statistical parameters (reaction when a threshold is exceeded).
          \end{itemize}
\end{itemize}

An Active IDS always has someone who looks at IDS statistics. This is necessary because statistics may not be accurate. An attack is detected if it significantly diverges from the statistics related to the behavior of the system, so \ul{there is a chance to have undetected attacks but also false positives}. This means that human intervention is often needed to check the alarms raised by the IDS, carefully analyze the problem, and understand if there is an attack or if it is just a false positive, given the strict threshold. Typically, IDS have both an active and passive part.

\paragraph{Topological features of IDS:}

\begin{itemize}
    \item \textbf{HIDS (host-based IDS)}, looks for statistics inside a single node:
        \begin{itemize}
            \item Log analysis (OS, service, or application);
            \item Internal OS monitoring tools;
        \end{itemize}
    \item \textbf{NIDS (network-based IDS)}, looks at the network traffic:
        \begin{itemize}
            \item Network traffic monitoring tools.
        \end{itemize}
\end{itemize}

\subsection{NIDS}

\subsubsection{NIDS components}
\begin{itemize}
    \item \textbf{Sensor (host/network based):}
        \begin{itemize}
            \item Checks traffic and logs looking for suspect patterns;
            \item Then generates the relevant security events;
            \item Could also interact with the system (modifying ACLs, performing TCP reset, …);
        \end{itemize}
    
    \item \textbf{Director:}
        \begin{itemize}
            \item Coordinates the work of all sensors;
            \item Manages the security database (e.g., statistics, attack signatures);
        \end{itemize}
    
    \item \textbf{IDS message system (to make director and sensors communicate):}
        \begin{itemize}
            \item Provides secure (authentication and integrity) and reliable (nobody should be able to drop a message that was sent by sensors or vice versa) communication among the IDS components.
        \end{itemize}
\end{itemize}

To achieve reliability in communication, if possible, it is performed on a separate physical network. If it is not possible, a VLAN or VPN is used.


\subsubsection{NIDS architecture}
\begin{figure}[h]
    \centering
    \includegraphics[page = 35,trim = 1cm 2.2cm 1cm 4cm, clip, width = 0.64\textwidth]{\slides}
    \caption{NIDS architecture}
    \label{fig:nids-architecture}
\end{figure}

The schema (Figure \ref{fig:nids-architecture}) shows how IDS should be configured. Conceptually, there is always the three-legged firewall with a connection to the external, one to the internal, and one to the DMZ. In the internal network, \textbf{sensors} on the most critical hosts are inserted (e.g., databases, application servers). \textbf{Net sensors} are also needed to check if some malicious traffic succeeds in passing the network, but it is also necessary to check the behavior of users in the internal network.

On the DMZ, there is the server reachable from the outside and those are easily attackable. Therefore, a \textbf{host sensor} on each server placed on the DMZ should be inserted. The last sensor is the \textbf{net sensor} on the external interface of the firewall. Some think that it is not needed because it is outside, but placing it outside of the firewall permits the analysis of \underline{which attempts are made}, to gather statistics (e.g., the firewall blocks 98\% of attacks).

On the top right, the IDS director collects all the information from sensors.

\subsection{IPS - Intrusion Prevention System}

\textbf{IPS} is a system used to \textbf{accelerate and automate responses to intrusions}. It combines an \textit{IDS} with a \textit{ distributed dynamic firewall}. The IDS sends alarms related to a specific type of traffic; then, when the distributed dynamic firewall receives an alarm, it can block all traffic of that particular kind or isolate a node. This approach is effective when the intrusion detection system's detection is 100\% certain. However, if it is not, IPS can be risky because \textbf{it may make incorrect decisions and block innocent traffic}. It is a technology, not a product, with a significant impact on many elements of the protection system. 
It is often integrated into a single product called \textit{IDPS}.



\subsection{Next-Generation Firewall (NGFW)}
NGFW is a kind of firewall that attempts to \textbf{identify applications independently of the network port used}.
Consider a scenario where users are only allowed to use \texttt{HTTPS (443/TCP)}, and all other traffic is blocked. There might be someone using port 443 to send traffic other than HTTPS. \textbf{With a packet filter, it is not possible to detect that}. In contrast, with NGFW, it is possible to inspect traffic on any port and try to understand the actual application being carried out on that port. If possible, NGFW tries to decipher/re-cipher the traffic.

NGFW offers integration with authentication systems such as captive portal, 802.1x, or endpoint authentication (Kerberos, Active Directory, LDAP, ...). This integration allows the firewall to write policies based on user identities rather than IP addresses. By associating IP addresses with specific users through integration, it becomes possible to create \textbf{per-user policies}. Similarly, the ability to identify applications independent of the port used allows the application of \textbf{per-application policies}. Additionally, NGFW can perform filtering based on known vulnerabilities, threats, and malware.


\subsection{Unified Threat Management (UTM)}
It is an \textbf{integration of several products into a single device} that simplifies management. The device provided is named \textit{UTM appliance/Security} appliance and can contain firewall, VPN, anti-malware, content-inspection, IDPS, and other features. 

The actual capabilities depend on the manufacturer. UTM is mainly targeted to reduce the number of different systems, thus reducing management complexity and cost.

\subsection{Honey Pot / Honey Net}

\begin{figure}[h]
    \centering
    \includegraphics[page = 39,trim = .5cm 1.5cm 1.3cm 3.8cm, clip, width = 0.65\textwidth]{\slides}
    \label{Honey Pot / Honey Net}
\end{figure}

A Honey Pot is a method to \textbf{attract attackers}. The external network is protected with a firewall, which contains the real DMZ. Additionally, there is a second DMZ called the \textbf{decoy DMZ} where there is a \textbf{honey pot} — \textit{an easily attackable machine}. The purpose is to attract attackers and then have an inspection system that monitors what the attacker is doing to \textbf{understand their behavior} and new types of attacks or to collect malware. Honey pots are placed worldwide by antimalware manufacturers just to collect new malware.


The honey pot can also be placed inside the internal network to detect if there are internal users attempting to attack the system. For example, it is possible to create a fake duplicate salary server just to check if someone is trying to increase their own salary or to access someone else’s salary.
