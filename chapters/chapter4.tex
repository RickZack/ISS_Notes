\chapter{Security of IP networks}

\begin{figure}[h]
    \centering
    \includegraphics[page = 2,trim = 0.7cm 2.2cm 0.7cm 4cm, clip, width = 0.55\textwidth]{\slides}
    \caption{Remote access via dial-up lines}
\end{figure}

%\paragraph*{Remote access via dial-up lines}
The first point regarding the security of IP networks involves \ul{controlling who can access the networks}. 
In the past, it was primarily used for residential users who utilized a modem to connect to a telephone line (switched network). 
This process involved transforming bits into sound and having equivalent equipment on the ISP side, which would accept the telephone call and convert it into network packets. 
To facilitate this, devices known as \textbf{NAS} (\textit{Network Access Server}) were employed. 
NAS had the responsibility of \underline{authenticating users}, \underline{performing access control}, and subsequently \underline{providing users with access} to the IP network - typically the internet, but in some cases also allowing access to a company's internal network from home. 
Nowadays, this system is no longer widely used, at least in Western countries, although some countries still rely on it.


\begin{figure}[h]
    \centering
    \includegraphics[page = 3,trim = 0.7cm 2.2cm 0.7cm 3.7cm, clip, width = 0.55\textwidth]{\slides}
    \caption{Network access (modern way)}
\end{figure}

%\paragraph*{Network access (modern way)}
Today, it is possible to access the Internet in different ways, which basically depends on the device. 
For example, a smartphone typically uses technologies such as 3G, 4G, or 5G to connect to the base station, which runs an \textbf{authentication protocol} with the \textbf{AAA server} to check if the user is authorized for an internet connection. 
It is also possible to use Wi-Fi to connect to access points, which provide the translation from a wireless to a wired network and can again verify access. 
Alternatively, there could be a home gateway, which provides access to the Internet using both Ethernet and Wi-Fi (only if properly authenticated). 
In all these scenarios, authentication is always required before permitting traffic from a specific user. 
Therefore, there is a difference between the \textit{local branch} / \textit{last mile}, \textit{border element}, and \textit{the core network}.



\section{Authentication of PPP channels}
