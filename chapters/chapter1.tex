\chapter{Chapter}

\section{Introduction to the security of ICT systems}
Cybersecurity has become very important in today's world. Since every system relies on computer systems, any kind of damage can result in significant economic losses. Even indirect attacks that do not aim to steal money have economic costs. Cybersecurity is essential because attacks can be performed without the need to physically access the target location.


The reasons why cybersecurity is important are as follows:
\begin{itemize}
\item Big damage on successful attacks
\item Easy accessibility of systems
\end{itemize}

We must consider all the possible consequences of a successful attack. First, there can be \textbf{financial loss} (direct loss, for example, if someone gains access to bank account credentials, and indirect loss if the revelation that the company has been attacked negatively affects the stock exchange). There can be \textbf{recovery costs} because every successful attack results in damage, and there will be expenses required to return the system to normal operations and to enhance it to prevent new attacks. There can also be \textbf{productivity losses} if the attacks halt or delay processes. A successful attack may lead to \textbf{business disruption} because customers may seek alternative suppliers if a company is vulnerable to attacks.


For all these reasons we should protect systems. Most of the innovation is based on two main pillars:
\begin{itemize}
    \item The ability to communicate from any part of the world (communication networks)
    \item The increasing use of personal and mobile devices
\end{itemize}
These two foundations are no longer sufficient for innovative products; every new product now requires a security system.


\section{Complexity of the ICT scenario}
The ICT scenario is complex for various reasons. One key factor is the sheer number of different mobile and connected \textbf{devices}, including desktops, laptops, tablets, smartphones, smart TVs, fridges, and cars. All these devices can now connect to the internet, making security a critical concern.
\textbf{Communication networks } have shifted to data-only networks, meaning there are no more analog phone networks. This change implies that almost everything is vulnerable to potential attacks. It's not just wireless networks that can be targeted; even wired networks are susceptible to security threats.
\textbf{Distributed services} are on the rise, requiring constant technical solutions to keep them running. This often involves outsourcing parts of server management, hosting, and adopting cloud services. This means that computers are no longer confined within a company, which necessitates trust in the service providers. Additionally, software development is getting more complicated due to various factors like software layering, framework integration, and the use of multiple programming languages. This complexity increases the chances of errors and vulnerabilities.
In terms of security, the challenges can be summarized by the first engineering axiom:
\begin{center}
    "The more complex a system is, the harder it is to ensure its correctness."    
\end{center}
Therefore, it's essential to keep systems as simple as possible. For instance, the number of bugs in a program tends to increase more than proportionally with the number of lines of code. The current complexity of information systems favors attackers, who can discover increasingly ingenious and unforeseen attack paths.\\ 
To express this idea clearly, we follow the \textbf{KISS rule}: \emph{"Keep it Simple, Stupid."}




% TMP old only
\ifthenelse{\boolean{showOld}}{
  \section{A definition of ICT Security}
  Each of us has a different concept of security: for example, the obligation to use safety belts when driving depends on country to country. Security is a personal concept but in engineering, we need to provide some formal definitions.
  Cybersecurity is a distributed part of a company, and every employee of a company must have the awareness for cybersecurity.
  \begin{enumerate}
    \item[1.] "Cybersecurity is the set of products, services, organization rules, and individual behaviors that protect the ICT system of a company."
  \end{enumerate}
  Let us explain the keywords in the definition.
  \begin{description}
    \item [Products:] refers to something that people can buy (such as products for firewall and VPN);
    \item [Services:] these services are implemented by buying products;
    \item [Organization rules:] they are required because even if, for example, a new system is set up with a password, rules must provide information to employees on how complex the password must be; otherwise, there will be no rules but personal behaviors, which could make the use of technical solutions less effective.
  \end{description}
  \begin{enumerate}
    \item [2.] It is the duty to protect the resources from undesired access, guarantee the privacy of information, ensure the service operation, and availability in case of unpredictable events (\textbf{C.I.A. = Confidentiality, Integrity, Availability}).
  \end{enumerate}

  More in detail, these three concepts have the following meaning:
  \begin{itemize}
    \item \textbf{Confidentiality} covers two related concepts:
    \begin{description}
        \item [Data confidentiality:] assures that private or confidential information is not made available or disclosed to unauthorized individuals.
        \item [Privacy:] assures that individuals control or influence what information related to them may be collected and stored and by whom and to whom that information may be disclosed.
    \end{description}
    In terms of requirements and the definition of a loss of security, it means preserving authorized restrictions on information access and disclosure, including means for protecting personal privacy and proprietary information. A loss of confidentiality is the unauthorized disclosure of information.

    \item \textbf{Integrity covers} two related concepts:
    \begin{description}
        \item [Data integrity:] assures that information (both stored and in transmitted packets) and programs are changed only in a specified and authorized manner.
        \item [System integrity:] assures that a system performs its intended function in an unimpaired manner, free from deliberate or inadvertent unauthorized manipulation of the system.
    \end{description}
    In terms of requirements and the definition of a loss of security, it means guarding against improper information modification or destruction, including ensuring information non-repudiation and authenticity. A loss of integrity is the unauthorized modification or destruction of information.

    \item \textbf{Availability:} assures that systems work promptly, and service is not denied to authorized users. In terms of requirements and the definition of a loss of security, it means ensuring timely and reliable access to and use of information. A loss of availability is the disruption of access to or use of information or an information system.
  \end{itemize}

  This definition is a good starting point for cybersecurity, but more than the C.I.A. triad is required; two of the most mentioned are as follows:

  \begin{itemize}
    \item \textbf{Authenticity:} the property of being genuine and being able to be verified and trusted; confidence in the validity of a transmission, a message, or message originator. This means verifying that users are who they say they are and that each input arriving at the system came from a trusted source.

    \item \textbf{Accountability:} the security goal that generates the requirement for actions of an entity to be traced uniquely to that entity. This supports non-repudiation, deterrence, fault isolation, intrusion detection and prevention, and after-action recovery and legal action.
  \end{itemize}

  \begin{enumerate}
    \item[1.] "The objective is to guard the information with the same professionalism and attention as for the jewels and deposit certificates stored in a bank caveau".
    \item[3.] "The ICT system is the safe of our most valuable information; ICT security is the equivalent of the locks, combinations, and keys required to protect it".
  \end{enumerate}

}{
  % else
}



\section{Risk estimation}

\begin{wrapfigure}{r}{0.55\textwidth}
  \centering
  \includegraphics[page=8, width = 0.55\textwidth]{\slides}
\end{wrapfigure}

Before setting up a defense, we must understand what the risks are. To make a risk estimation, it is good to start from the \textbf{service}. Once we know the service we need to protect, we must identify the assets used to provide that service, and there are four categories of assets: \textbf{ICT resources} (computers, disks, networks), \textbf{data} (not the disks but something intangible that could be deleted or modified), \textbf{location} (assets must be inside a protected room), and \textbf{human resources} (which means the group of people who possess the knowledge that must not be shared).

After considering the assets, the next step is to identify the events that could affect their normal operation. The first point is that each asset has some \textbf{vulnerabilities} (for example, disks that are vulnerable to physical damage like a hammer hitting the disk), and some vulnerabilities can pose a real \textbf{threat} depending on the environment. For example, if a disk is left in an open place, someone might use a hammer to damage the disk. However, if the disk is locked in a room where nobody can access it, the vulnerability still exists but is not a real threat.

\vspace{5mm}
So, the process of analyzing a service searching for risks take place as follows:
\begin{itemize}[]
  \item Find the \textbf{assets} of the service to be protected;
  \item Finding the \textbf{vulnerabilities} of each asset;
  \item Finding the \textbf{threats}, giving the way in witch the assets are used;
\end{itemize}

\vspace{5mm}
Once we identify the threats we must:
\begin{itemize}
  \item decide for each threat which \textbf{impact} it could have (what happens if disk is destroyed? If there's only one copy it could be a disaster, but if there are many that's not a problem)
  \item the \textbf{event probability} of the threat. This is the last point to get the \textbf{risk estimation}.
\end{itemize}

\bigskip
Recap of terminology:
\begin{description}
  \item[Asset:] the set of goods
  \item[Vulnerability:] weakness of an asset; 
  \item[Threat:] deliberate action/accidental even that can procure the loss of a security property exploiting a vulnerability;
  \item[Attack:] threat occurrence (deliberate action);
  \item[(negative) Event:] threat occurrence     
\end{description}


% TMP: new content
\ifthenelse{\boolean{showNew}}{
  \section{Risk management}
  To prioritize the risks, we can build a \textbf{risk assessment matrix} (or risk heat map):
  \begin{figure}[h]
    \includegraphics[page = 10,trim = 1cm 3cm 1cm 12cm, clip, width = \textwidth]{\slides}
  \end{figure}
} {}
% end TMP: new content


\section{Analysis and management of security}

\begin{wrapfigure}{l}{0.40\textwidth}
  \centering
  \includegraphics[page = 11,trim = 2cm 2.1cm 2cm 4cm, clip, width = 0.40\textwidth]{\slides}
\end{wrapfigure}
Risk estimation is only a piece for analysis and management of security. We can see that \textbf{assets}, \textbf{vulnerabilities}, \textbf{threats} give us the \textbf{risks} and then we start the \textbf{management} of the security. This means that for those risks that are not acceptable, either because they have high impact or high probability to end up in an attack, we need to select countermeasures and implement them. The last step is \textbf{audit}, that means that some independent person comes to check our work (if we correctly identified risks, selected the correct countermeasures, implemented them correctly).

In this course we will consider three of these blocks: vulnerabilities, the available countermeasures, and how to implement countermeasures.


\begin{wrapfigure}{o}{0.55\textwidth}
  \centering
  \includegraphics[page = 12,trim = 0.4cm 3.5cm 0.8cm 4cm, clip, width = 0.55\textwidth]{\slides}
\end{wrapfigure}
What is the correct step in the lifecycle of a system to implement security?\\
In brief, there is no single correct point for implementing security because it must be addressed at each stage of the design process.\\ 
In more detail, when we perform the \textbf{analysis of requirements} for our system, we must conduct a risk assessment; based on these risks, we can define security policies and procedures that we will apply throughout the rest of the system design.\\ 
When evaluating \textbf{technical options}, we also need to identify security products. For example, when choosing a database, we must consider security alongside other factors such as speed and cost. If we opt for a database that automatically encrypts data, we have already addressed a security concern. Conversely, if we choose a faster database that lacks an encryption system, we will need to design that separately, incurring additional costs and efforts. \\
When \textbf{designing} the services that the system will offer, we must also include the security services component. Security should be integrated at each stage of the design process, not added as an afterthought. If we create a prototype website or app without any security measures, it will be challenging to retrofit security later on.


\section{Relation in the security field}
During the development of our system, we must integrate security at each step and ensure that the design is correctly implemented. We must test the system, including its security aspects. An example of this is testing against unexpected inputs to ensure that the system does not accept and process incorrect inputs.

While implementing our system, it is important to establish security mechanisms. Some systems include security functions, but they are often deactivated due to associated costs (generally, activating security may slow down the system). For instance, the default password of a home router may be shared among different units, so it is necessary to change that password.

When the system is operational, security must be managed on a daily basis, as the security landscape changes constantly.


\section{Relation in the security field}

The black box is the system itself.
Assets are exposed to vulnerabilities, and those vulnerabilities increase security risks. 
Additionally, threats exploit vulnerabilities, and the existence of vulnerabilities creates the opportunity for a threat.
\begin{wrapfigure}{l}{0.55\textwidth}
  \centering
      \includegraphics[page = 14,trim = 1cm 3cm 1cm 5cm, clip, width = 0.55\textwidth]{\slides}
  \end{wrapfigure}
On the left, there are the security requirements that we want to implement. Security requirements are indicated by security risks, and security requirements are met by security controls.

The \textbf{security control} is the most important piece of the picture nowadays. It is an element placed in the system to protect against a specific threat and reduce the risks to which the system is exposed. Examples of security controls include firewalls, VPNs, and disk encryption.

\bigskip
Some terminology:
\begin{description}
  \item[Incident:] A security event that compromises the integrity, confidentiality, or availability of an information asset (generic definition).
  \item[(Data) breach:] An incident that results in the disclosure or potential exposure of data.
    \begin{itemize}
      \item Disclosure: Occurs when data is intentionally given to someone.
      \item Exposure: Data becomes available to anyone who knows where to find it.
    \end{itemize}
  \item[(Data) disclosure:] A breach for which it is confirmed that data was actually disclosed (not just exposed) to an unauthorized party.
\end{description}
The difference between the last two is that the last one is a breach involving data that were not just exposed but also confirmed to be disclosed to an unauthorized party.


\newpage
\section{Window of exposure}
\begin{wrapfigure}{r}{0.60\textwidth}
\centering
    \includegraphics[page = 16,trim = 0.2cm 1.7cm 0.2cm 4cm, clip, width = 0.60\textwidth]{\slides}
\end{wrapfigure}
The \textbf{window of exposure} is \emph{the time between the discovery of a new vulnerability and the installation of a patch}.


Analyzing the graph from the left, we notice a consistently low level of risk, which, although it can never be reduced to 0, remains close to it.
However, at a certain juncture, a new vulnerability is discovered, causing the risk to surge due to its uncontrollable nature.
At a specific point (marked as the red point), an individual exploits this vulnerability to execute an attack. This action leads to the vulnerability becoming public, thereby making it accessible to everyone.
This initial stage is commonly referred to as \textbf{discovery}.

After an attack is carried out, two distinct groups of individuals become aware of it: the \emph{bad guys} who aim to compromise the system and the \emph{good guys} who strive to safeguard the system. Within these two categories, there is also the product vendor, who must be informed of the vulnerability. The vendor should, in turn, promptly notify its customers of the newly discovered vulnerability. While the vendor is working on fixing the vulnerability, users of that product should refrain from attempting to fix it, which is typically impossible. Instead, they should focus on updating their security tools, at the very least, to detect if the vulnerability is actively being exploited in attacks. This is the \textbf{publication} phase during which the vulnerability becomes public knowledge, and everyone is awaiting a fix while also making efforts to detect potential attacks.

Finally, at some point in time, the vendor creates a patch to fix the vulnerability. However, the patch must be distributed, and the risk only decreases when the patch becomes widely known and is eventually installed, rendering the system \textbf{protected}. The window of exposure can persist for days or even months, which is why security is an ongoing effort.

\section{Some statistics}
\begin{wrapfigure}{r}{0.55\textwidth}
  \centering
  \includegraphics[page = 17,trim = 1cm 2.6cm 1cm 7cm, clip, width = 0.55\textwidth]{\slides}
\end{wrapfigure}
The graphic shows that there are about 10 million attacks per month using malware. That is why we should always keep our anti-virus/anti-malware updated.
If we consider web servers across all sectors, 44\% of servers were consistently vulnerable to attacks every day of the year.
\begin{figure}[h]
  \centering
  \includegraphics[page = 18,trim = 1cm 1.7cm 1cm 7cm, clip, width = 0.60\textwidth]{\slides}
\end{figure}

The term \textbf{0 day} refers to the first day when a vulnerability is reported to the public and has not yet been fixed.

The banking sector boasts the best security track record, with the lowest vulnerability rate (rarely vulnerable for 30 days or less each year).

In general, achieving strong security is challenging. Vulnerabilities can be discovered by both well-intentioned individuals (good guys) and malicious actors (bad guys). Some people investigate software to uncover vulnerabilities and inform the vendor to patch them before they are exploited by malicious actors. This proactive approach is taken to ensure patches are in place before vulnerabilities are discovered by the bad guys.

The "\textbf{0-day initiative}" (\textbf{ZDI}) discovers vulnerabilities and notifies the relevant organizations \emph{before} making them public. ZDI typically provides a 120-day grace period from the discovery of a new vulnerability to allow the vendor to fix it before disclosing it to the public. Longer deadlines can be risky, as bad guys may discover the vulnerability during the extended time frame.

\vspace{5mm}
Example:
\begin{itemize}
  \item 8 May 18: ZDI reports the vulnerability to the vendor, and the vendor acknowledges the report.
  \item 14 May 18: The vendor replies that they have successfully reproduced the issue reported by ZDI.
  \item 9 Sep 18: The vendor reports an issue with the fix and states that the fix might not be included in the September release.
  \item 10 Sep 18: ZDI issues a caution about a potential 0-day (which means that the vulnerability, for which a fix is not available, is going to be published).
  \item 11 Sep 18: The vendor confirms that the fix did not make it into the build.
  \item 12 Sep 18: ZDI confirms its intention to 0-day on 20 Sep 18.
\end{itemize}

% TMP: new part
\ifthenelse{\boolean{showNew}}{
  \section{Cyber threats}

  \subsection*{Components}
  \begin{wrapfigure}{r}{0.30\textwidth}
    \centering
        \includegraphics[page = 20,trim = 8cm 3cm 8cm 11cm, clip, width = 0.30\textwidth]{\slides}
    \end{wrapfigure}
  There are \textbf{three main components} in cyber threats:
  \begin{itemize}
    \item threats actors (and their motivation)
    \item attack vectors (vulnerabilities and context)
    \item vulnerable targets (value for owner and attacker)
  \end{itemize}


  \subsection*{Motivations: \emph{MICE}}
  What are the motivations behind this?
  \begin{itemize}
    \item \textbf{M is for Money:} direct transfer, blackmail, ... or indirect (e.g. data reselling);
    \item \textbf{Is is for Ideology:} political, religious, hacktivism;
    \item \textbf{C is for Compromise:} individuals with no choice due to blackmail or threat against their families or themselves;
    \item \textbf{E is for Ego:} bragging around and positive reputation, "we do it because we can".
  \end{itemize}
  \begin{figure}[h]
    \includegraphics[page = 21,trim = 22cm 2.1cm 1cm 12cm, clip, width = 0.10\textwidth]{\slides}
  \end{figure}
  \begin{figure}[h]
    \centering
    \includegraphics[page = 22,trim = 1cm 2.1cm 1cm 4cm, clip, width = 0.50\textwidth]{\slides}
    \caption{Threat actors}
  \end{figure}


  \subsection*{Standardization bodies}
  Standardization bodies in cybersecurity are organizations that develop and publish standards and guidelines to enhance security and interoperability in computer systems and networks.
  %TODO: might be better to add subfigures here
  \begin{figure}[!h]
    \centering
    \includegraphics[page = 23,trim = 1cm 3cm 1cm 2cm, clip, width = 0.50\textwidth]{\slides}
    \includegraphics[page = 24,trim = 1cm 3cm 1cm 2cm, clip, width = 0.50\textwidth]{\slides}
  \end{figure}
} {}




\section{What is Security?}
\emph{"Security is a process, not a product".}
\begin{flushright}
  (Bruce Schneier, Crypto-Gram, May 2005)
\end{flushright}
If we have learned anything from the past couple of years, it is that \textbf{computer security flaws are inevitable}.
Systems break, vulnerabilities are reported in the press, and still many people put their faith in the next product, or the next upgrade, or the next patch. "This time it's secure," they say. So far, it has not been.
\textbf{Security is a process, not a product}. Products provide some protection, but the only way to effectively do business in an insecure world is to put processes in place that recognize the inherent insecurity in the products.
\textbf{The trick is to reduce your risk of exposure regardless of the products or patches}.

\subsection*{Security Principles}
To make it possible, it is necessary to follow some security principles:
\begin{itemize}
  \item \textbf{Security in depth}: if the enemy can defeat the first line of defense, there must be a second line to stop the attacker. Do not rely on just one defense, as that defense may have a bug or problem. It is better to have multiple levels of defense, so as the attacker breaks through the defenses, it will become increasingly difficult to keep penetrating.
  
  \item \textbf{Security by design}: this means that the security design is integrated into the system from the beginning and not added as an afterthought.
  
  \item \textbf{Security by default}: users should not have the choice to activate or deactivate security. Security should be enabled by default, and it should require significant effort to disable security features.
  
  \item \textbf{Least privilege}: this principle dictates that any element operating within the system should be assigned the minimum amount of privileges necessary to perform its task. Imagine a scenario where a system has excessive privileges, and it is being attacked by a virus. The virus could gain access to everything because of the excessive privileges.
  
  \item \textbf{Need-to-know}: this principle emphasizes that access to any component of the system should be granted only for the data required to execute a specific task. For example, in the case of Amazon, several people work within the system. When a customer places an order on Amazon, the first person handling the order can only see the details of what was ordered. They do not have access to information about who placed the order or the destination of the goods.
\end{itemize}

% TMP: old content1
\ifthenelse{\boolean{showOld}}{
\subsection*{European Central Bank}
ECB made on 31/01/2013 some "\emph{Recommendations for the security of Internet payments}". In general, when there is a regulation there are \textbf{generic} recommendations, but typically lawyers do not want to get into details. In the security scenario this is changed recently. It is not possible to let individual companies decide what are the good rules for security. These recommendations apply to payment schemas governance authorities, payment service providers (PSP), and merchants that use credit cards for payments.

The main recommendations are:
\begin{itemize}
    \item Protect the initiation of Internet payments, as well as access to sensitive payment data, by \textbf{strong customer authentication};
    \item \textbf{Limit the number of log-in or authentication attempts}, define rules for Internet payment services session “time out,” and set time limits for the validity of authentication;
    \item Establish \textbf{transaction monitoring mechanisms} designed to prevent, detect, and block fraudulent payment transactions (reminiscent of Security in depth);
    \item \textbf{Provide assistance} and guidance to customers about best online security practices, set up alerts, and provide tools to help customers monitor transactions.
\end{itemize}
}{}

\subsection*{Security properties}
\begin{center}
  \begin{tabular}{||l l||}
    \hline
     Authentication (simple/mutual) & autenticazione (semplice/mutua) \\
     Peer authentication & autenticazione (della controparte) \\
     Data/origin authentication & autenticazione (dei dati) \\ 
     Authorization, access control & autorizzazione, controllo accessi \\
     \textbf{Integrity} & \textbf{integrità}\\
     \textbf{Confidentiality, privacy, secrecy} & \textbf{riservatezza, confindenzialità} \\ 
     Non-repudiation & non ripudio \\ 
     \textbf{Availability} & \textbf{disponibilità}\\
     Traceability, accountability  & tracciabilità \\
    % \hline
  \end{tabular}
\end{center}



